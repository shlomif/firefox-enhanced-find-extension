% LaTeX file for a 1 page document
\documentclass{acm_proc_article-sp}

\title{SmartFind: Type allowing errors in small devices\\or\\A pragmatic use of Levenshtein in Web-Content pages in mobile screen}

\author{Roberto Santos \and Allan Bezerra \and Antonio Gomes \and Tomaz Noleto \and Andre Pedralho}

\begin{document}
\maketitle

\begin{abstract}

one way to
\end{abstract}

\section{Introduction}


Small devices are completely common in our needs during the day for many reasons, make call, browse web pages, read e-amils and etc.
It is very habitual that Business men, for example, need use the mobile devices to read e-mails or browse web pages to view news in large pages (with many words).
Supponsing a very common scenario, when the man is in a page with many words and it's necessary to go to a specific chunk of the page, normally a command to locate a term in this page is used. In this case, the chars contained in the term, obaying the order where they occurs, are typed and the command goes to locate this term.
One problem, in this scenario, happens when it's necessary find a term that tha correct ortograph is not known. This can happnens for many reasons: keyboard layout is not correct, the user is in a page that is not familiarized with the words

\section {Experiments}
Idioma.

livre de acentua��o.

navega��o por contexto.

\end{document}
